
\usepackage [ a4paper, ignorehead, ignorefoot, ignoremp,
asymmetric % , showframe
, twoside, pdftex , papersize={17cm,24cm} , left = 2.50cm, right =
1.50cm , top = 2.70cm, bottom = 2.70cm , headsep = 0.5cm, footskip =
1.0cm , marginparsep = 0.3cm, marginparwidth = 1.2cm ]{geometry}


% Paquetes
\usepackage[utf8]{inputenc} \usepackage[T1]{fontenc}
\usepackage{lmodern} \usepackage[scaled=.85]{helvet}
\usepackage{times} \usepackage{xspace,calc} \usepackage{url}
\usepackage{fancyhdr} \usepackage{graphicx}
\usepackage[spanish,english]{babel} \usepackage{latexsym}
\usepackage{amsmath,amssymb,mathrsfs,stmaryrd} \usepackage{amsthm}
\usepackage{semantic} \usepackage{multirow}
\usepackage{float}
\usepackage{chngpage}    
\usepackage{ipa} \usepackage[usenames,dvipsnames]{xcolor}
		
\usepackage{epigraph} \renewcommand{\epigraphrule}{0pt}
		
\usepackage[title, titletoc]{appendix}
    
\addto\captionsenglish{% Replace "english" with the language you use
  \renewcommand{\contentsname}%
  {Contenido}%
}
    
\addto\captionsenglish{%
  \renewcommand\appendixname{Anexo}
  \renewcommand\appendixpagename{Anexos} }
		
\usepackage{tocloft} \setlength{\cftbeforechapskip}{6pt}
\tocloftpagestyle{empty}
		
\usepackage[all]{xy} \usepackage{caption} \usepackage{subcaption}

\usepackage[explicit]{titlesec} \usepackage{lipsum}

\titlespacing*{\section}{0pt}{\baselineskip}{\baselineskip}
\titlespacing*{\subsection}{0pt}{\baselineskip}{\baselineskip}
\titlespacing*{\subsubsection}{0pt}{\baselineskip}{0.5\baselineskip}

\usepackage[unicode=true,colorlinks,urlcolor=CadetBlue,linkcolor=BlueViolet,citecolor=BlueViolet,linktocpage=true]{hyperref}

\newcommand{\TODO}[1]{{\color{red}ToDo: #1}}
		
\newcommand{\red}[1]{{\color{red}#1}}

\newlength\chapnumb \setlength\chapnumb{4cm}

\titleformat{\chapter}[block] {\normalfont\sffamily}{}{0pt}
{\parbox[b]{\chapnumb}{%
    \fontsize{120}{110}\selectfont\thechapter}%
  \parbox[b]{\dimexpr\textwidth-\chapnumb\relax}{%
    \raggedleft%
    \hfill{\LARGE#1}\\
    \rule{\dimexpr\textwidth-\chapnumb\relax}{0.4pt}}}
				
\titleformat{name=\chapter,numberless}[block]
{\normalfont\sffamily}{}{0pt} {\parbox[b]{\chapnumb}{%
    \mbox{}}%
  \parbox[b]{\dimexpr\textwidth-\chapnumb\relax}{%
    \raggedleft%
    \hfill{\LARGE#1}\\
    \rule{\dimexpr\textwidth-\chapnumb\relax}{0.4pt}}}
			
% Colored Boxes
\newlength{\savedparindent}
\newcommand{\SaveIndent}{\setlength{\savedparindent}{\parindent}}
\newcommand{\RestoreIndent}{\setlength{\parindent}{\savedparindent}}

\newcommand{\InGray}[1]{ \SaveIndent{}
  % \noindent{}
  \fcolorbox[rgb]{0,0,0}{0.95,0.95,0.95}{
    \begin{minipage}{0.965\linewidth}
      \RestoreIndent{}%
      #1
    \end{minipage}
  } }

\newcommand{\Review}[1]{ \SaveIndent{}
  % \noindent{}
  \fcolorbox[rgb]{0,0,0}{0.88,0.88,0.88}{
    \begin{minipage}{0.965\linewidth}
      \textbf{Review:}\\
      \RestoreIndent{}%
      #1
    \end{minipage}
  } }
		
% Figures
\newcommand{\tablefigure}[3]{
  \begin{figure}[ht]
    \center \rule{\textwidth}{0.5pt}\\[0.4cm]
    \begin{tabular}{l}
      \input{#1}
    \end{tabular}\\[0.4cm]
    \center \rule{\textwidth}{0.5pt}
    \centering \caption{\label{#2} #3}
  \end{figure}
}

\newcommand{\tablepicture}[4]{
  \begin{figure}[ht]
    \begin{center}
      \rule{\textwidth}{0.5pt}\\[0.4cm]
      \includegraphics[width=#1,keepaspectratio]{#2}\\[0.4cm]
      \rule{\textwidth}{0.5pt}
      \caption{\label{#3} #4}
    \end{center}
  \end{figure}
}

\newcommand{\tablepicturerotate}[5]{
  \begin{figure}[ht]
    \begin{center}
      \rule{\textwidth}{0.5pt}\\[0.4cm]
      \includegraphics[angle=#1,width=#2,keepaspectratio]{#3}\\[0.4cm]
      \rule{\textwidth}{0.5pt}
      \caption{\label{#4} #5}
    \end{center}
  \end{figure}
}

% Header and Footer
\pagestyle{fancy} \fancyhf{} \lhead{\nouppercase{\textsl{\rightmark}}}
\rhead{\nouppercase{\textsl{\leftmark}}} \fancyhead[LE,RO]{\thepage}

% Paragraph
\setlength{\parindent}{0cm} % sirve para que no indente por defecto los párrafos
\setlength{\parskip}{8pt} % sirve para dejar un espacio entre párrafos

% Hyphenation
\hyphenation{a-gra-de-ci-mien-to mo-de-los pro-ble-ma}

\fussy % sirve para que no te acalambre con los underfull y overfull
\usepackage{enumitem,amssymb}
\newlist{todolist}{itemize}{2}
\setlist[todolist]{label=$\square$}
\usepackage{pifont}
\newcommand{\cmark}{\ding{51}}%
\newcommand{\xmark}{\ding{55}}%
\newcommand{\done}{\rlap{$\square$}{\raisebox{2pt}{\large\hspace{1pt}\cmark}}%
\hspace{-2.5pt}}
\newcommand{\wontfix}{\rlap{$\square$}{\large\hspace{1pt}\xmark}}
